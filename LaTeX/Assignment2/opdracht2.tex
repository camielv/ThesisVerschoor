\documentclass{article}
\usepackage{hyperref}

\title{Opdracht 2\\\large Afstudeerproject B.Sc. Kunstmatige Intelligentie\\Faculteit der Natuurwetenschappen, Wiskunde en Informatica\\
Universiteit van Amsterdam}
\date{\today}
\author{C.R. Verschoor\\\href{mailto:Verschoor@uva.nl}{Verschoor@uva.nl}\\UvAnetID: 6229298\\StudentID: 10017321}
\begin{document}

\maketitle

\section{Opdracht 1}
\subsection*{Beschrijving}
Bekijk de laatste 5 artikelen uit \href{http://www.jair.org/vol/vol43.html}{JAIR volume 43}. Geef voor elk artikel in het kort:
\begin{description}
\item[Vraagstelling / hypothese] (max. 3 zinnen).
\item[Conclusie / claim] (max. 3 zinnen).
\item[Type vraagstelling] empirisch, methode, formeel, science of artificial (kies er 1 of formuleer combinatie)
\item[Type onderzoek/onderbouwing] bv. methode implementeren en testen, stelling bewijzen.
\end{description}

\subsection{Artikel 1}
\begin{description}
\item[Titel] Proximity-Based Non-uniform Abstractions for Approximate Planning.
\item[Vraagstelling / hypothese] 
\item[Conclusie / claim] 
\item[Type vraagstelling]
\item[Type onderzoek/onderbouwing]
\end{description}

\subsection{Artikel 2}
\begin{description}
\item[Titel] Avoiding and Escaping Depressions in Real-Time Heuristic Search.
\item[Vraagstelling / hypothese] 
\item[Conclusie / claim] 
\item[Type vraagstelling]
\item[Type onderzoek/onderbouwing]
\end{description}

\subsection{Artikel 3}
\begin{description}
\item[Titel] Reformulating the Situation Calculus and the Event Calculus in the General Theory of Stable Models and in Answer Set Programming.
\item[Vraagstelling / hypothese] 
\item[Conclusie / claim] 
\item[Type vraagstelling]
\item[Type onderzoek/onderbouwing]
\end{description}

\subsection{Artikel 4}
\begin{description}
\item[Titel] A Market-Inspired Approach for Intersection Management in Urban Road Traffic Networks.
\item[Vraagstelling / hypothese] Hoe moet een verkeersysteem eruit zien om de regelmatige verkeersopstoppingen in een stedelijk wegennet op te lossen.
\item[Conclusie / claim]  Een homogene distributie van voertuigen over het wegennet leidt tot een beter gebruik van het wegennet. Dit resulteert in kortere reistijden.
\item[Type vraagstelling] Methode omdat ze een voorstel willen doen om het probleem op te lossen en daarvoor testen verschillende methodes.
\item[Type onderzoek/onderbouwing] Methodes implementeren en testen.
\end{description}

\subsection{Artikel 5}
\begin{description}
\item[Titel] Learning to Win by Reading Manuals in a Monte-Carlo Framework.
\item[Vraagstelling / hypothese] Helpen handleidingen en gidsen de computer (controlerende applicatie) om in situaties, situatie-passende acties te selecteren. Hoeveel werkt deze beter dan een computer die leert op basis van de gamescore.
\item[Conclusie / claim] Door een goede koppeling tussen controle en taalkundige kenmerken is het model in staat om krachtige prestaties te leveren in aanwezigheid van de ruis van automatische taalanalyse.
\item[Type vraagstelling] Methode en Empirisch omdat het experiment nog nooit gedaan is en dit de eerste test er van is. Methode omdat er nog niet echt een methode was hiervoor.
\item[Type onderzoek/onderbouwing] Methode implementeren en testen.
\end{description}

\section{Opgave 2}
\subsection*{Beschrijving}
Bedenk beoordelingscriteria voor een artikel. Maak een beoordelingsformulier in de vorm van lijst van beoordelingscriteria voor AI onderzoek.
\subsection{Beoordelingscriteria}
\begin{description}
\item[Onderzoeksvraag] Duidelijk en overtuigend; orgineel, significant, inzichtelijk.
\item[Organisatie] Logische volgorde, creatieve organisatie, natuurlijke opbouw op basis van de onderzoeksvraag en inhoud.
\item[Ontwikkeling] De alineas zijn verbonden aan elkaar en lopen vloeiend over in elkaar. Goed gebruik van plaatjes en voorbeelden.
\item[Onderzoek/Informatievaardigheden] De bronnen zijn van goede kwaliteit en van goed aantal. Gebruik van diverse bronnen.
\item[Taalgebruik] Bewust van woordgebruik, inventief en creatief in woordkeuze.
\begin{description}
\item[Toon] gezaghebbend, eerlijk, bewust van publiek.
\item[Zinnen] Rijke variatie en complexiteit van zinstructuur.
\item[Mechanica] Beheerst de conventies
\end{description}
\end{description}

\section{Opgave 3}
\subsection*{Beschrijving}
Schrijf een complete (1 a 2 pag) review van `anonieme paper' en vul je eigen formulier in.

\end{document}